%%%%%%%%%%%%%%%%%%%%%%%%%%%%%%%%%%%%%%%%%%%%%%
% Big Data Coursework Report LaTeX Template for
% Faculty of Science and Engineering, University of Wolverhampton. 
% Compile By:
% Sujan Shrestha, Module Leader(shresthasujan2017@gmail.com)
%%%%%%%%%%%%%%%%%%%%%%%%%%%%%%%%%%%%%%%%%%%%%%%
\documentclass[conference]{IEEEtran}
\IEEEoverridecommandlockouts
% The preceding line is only needed to identify funding in the first footnote. If that is unneeded, please comment it out.
\usepackage{cite}
\usepackage{amsmath,amssymb,amsfonts}
\usepackage{algorithmic}
\usepackage{graphicx}
\usepackage{textcomp}
\usepackage{xcolor}
\def\BibTeX{{\rm B\kern-.05em{\sc i\kern-.025em b}\kern-.08em
    T\kern-.1667em\lower.7ex\hbox{E}\kern-.125emX}}
\begin{document}

\title{Big Data Analytics for Mitigating the Adverse Economic Impact of Covid-19 Lockdown Practices\\

\thanks{Identify applicable funding agency here. If none, delete this.}
}

\author{\IEEEauthorblockN{1\textsuperscript{st} Mr. Sujan Shrestha}
\IEEEauthorblockA{\textit{Faculty of Science and Engineering} \\
\textit{Herald College Kathmandu}\\
Kathmandu, Nepal \\
sujan.shrestha@heraldcollege.edu.np}
\and
\IEEEauthorblockN{2\textsuperscript{nd} Given Name Surname}
\IEEEauthorblockA{\textit{dept. name of organization} \\
\textit{name of organization}\\
City, Country \\
email address or ORCID}
}

\maketitle

\begin{abstract}
Abstract should have the following points:
-	Generic Information (1 Sentence)
-	Problem Statement(2 Sentences)
-	Aim/objective of the Report: (1 Sentence)
-	Contributions of the work connected with Methodology (2-3 Sentence)
-	Social/Research community Impact of the work (1 Sentence

\end{abstract}

\begin{IEEEkeywords}
Big Data Analytics, COVID-19, Economic Impact, Lockdown Practices, Machine Learning, Predictive Analytics, Resilience Strategies, Supply Chain
\end{IEEEkeywords}

\section{Background of the study}
* Generic Information (1 Paragraph)
* Problem Statement: Need to explain as a layman aspects, with statistical information and pictorial view. (1 Paragraph)
* Aim/objective of the work: (1 Paragraph)
* Contributions of the work connected with Methodology (should be itemised)
* Organization of the Report (1 Paragraph)


\section{Related Work}
This section discussed about the existing relevant works. You can discuss this section based on the category of the existing work and the final paragraph you should mention how your work differ with the existing works.
\begin{equation}
        O_{j}=\sigma(b+\sum_{i=1}^mW_{ij\cdot X_{i}}) \label{eq:ann}
    \end{equation}
Please number citations consecutively within brackets \cite{b1}. The 
sentence punctuation follows the bracket \cite{b2}. Refer simply to the reference 
number, as in \cite{b3}---do not use ``Ref. \cite{b3}'' or ``reference \cite{b3}'' except at 
the beginning of a sentence: ``Reference \cite{b3} was the first $\ldots$''

Number footnotes separately in superscripts. Place the actual footnote at 
the bottom of the column in which it was cited. Do not put footnotes in the 
abstract or reference list. Use letters for table footnotes.

Unless there are six authors or more give all authors' names; do not use 
``et al.''. Papers that have not been published, even if they have been 
submitted for publication, should be cited as ``unpublished'' \cite{b4}. Papers 
that have been accepted for publication should be cited as ``in press'' \cite{b5}. 
Capitalize only the first word in a paper title, except for proper nouns and 
element symbols.

For papers published in translation journals, please give the English 
citation first, followed by the original foreign-language citation \cite{b6}
\section{Methodology}
Here you should discuss the details methodology of your entire work process with a Block Diagram/Phases and later one discuss each block/phase in details. From Fig. \ref{fig:methodoclogy}
\begin{figure}[htbp]
\centerline{\includegraphics{Assets/Methodology.png}}
\caption{System Architecture Example}
\label{fig:methodoclogy}
\end{figure}
\section{Result and Discussion}
Should cover the experimental setup, discussion of the findings, and analysis of the findings with the below mentioned subtitle:
\subsection{Read In and Explore the Data}
\subsection{Data Analysis}
\subsection{Data Visualization}
\subsection{Cleaning Data}
\subsection{Choosing the Best Model}
\begin{table}[h!]
  \centering 
  \caption{\acrshort{ann} Model Evaluation for Test Data}
  \vspace*{5mm}
  \label{table:tab2}
  \begin{tabular}{ccccc}\hline
    \textbf{Inputs Layer} & \textbf{Accuracy} & \textbf{Precision} & \textbf{Recall} & \textbf{F1 Score}\\ \hline
    One Hot Encoding & 0.9975 & 0.9990 & 0.9960 & 0.9975\\
    Entity Embedding & 0.9961 & 0.9980 & 0.9939 & 0.9960\\
    Entity Embedding with PCA & \textbf{0.9980} & \textbf{0.9990}& \textbf{0.9970} & \textbf{0.9980}\\ \hline
  \end{tabular}
\end{table}
\section{Conclusion}
Highlight the overview of your work and the findings

\section*{References}
\begin{thebibliography}{00}
\bibitem{b1} G. Eason, B. Noble, and I. N. Sneddon, ``On certain integrals of Lipschitz-Hankel type involving products of Bessel functions,'' Phil. Trans. Roy. Soc. London, vol. A247, pp. 529--551, April 1955.
\bibitem{b2} J. Clerk Maxwell, A Treatise on Electricity and Magnetism, 3rd ed., vol. 2. Oxford: Clarendon, 1892, pp.68--73.
\bibitem{b3} I. S. Jacobs and C. P. Bean, ``Fine particles, thin films and exchange anisotropy,'' in Magnetism, vol. III, G. T. Rado and H. Suhl, Eds. New York: Academic, 1963, pp. 271--350.
\bibitem{b4} K. Elissa, ``Title of paper if known,'' unpublished.
\bibitem{b5} R. Nicole, ``Title of paper with only first word capitalized,'' J. Name Stand. Abbrev., in press.
\bibitem{b6} Y. Yorozu, M. Hirano, K. Oka, and Y. Tagawa, ``Electron spectroscopy studies on magneto-optical media and plastic substrate interface,'' IEEE Transl. J. Magn. Japan, vol. 2, pp. 740--741, August 1987 [Digests 9th Annual Conf. Magnetics Japan, p. 301, 1982].
\bibitem{b7} M. Young, The Technical Writer's Handbook. Mill Valley, CA: University Science, 1989.
\end{thebibliography}

\end{document}
